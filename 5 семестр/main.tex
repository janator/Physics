\documentclass[12pt]{kiarticle} % You can learn about my document class "kiarticle" and install it to your device by following the link: https://github.com/Kiarendil/toolkitex
\graphicspath{{pictures/}}
\DeclareGraphicsExtensions{.pdf,.png,.jpg,.eps}
%%%
\pagestyle{fancy}
\fancyhf{}
%\renewcommand{\headrulewidth}{ 0.1mm }
\renewcommand{\footrulewidth}{ .0em }
\fancyfoot[C]{\texttt{\textemdash~\thepage~\textemdash}}
\fancyhead[L]{Мюонный катализ \hfil}
\usepackage{multirow} % Слияние строк в таблице
\newcommand
{\un}[1]
{\ensuremath{\text{#1}}}
\newcommand{\eds}{\ensuremath{ \mathscr{E}}}
\usepackage{tikz}
%%% Работа с таблицами
\usepackage{array,tabularx,tabulary,booktabs} % Дополнительная работа с таблицами
\usepackage{longtable}  % Длинные таблицы
\usepackage{multirow} % Слияние строк в таблице

\usepackage[utf8]{inputenc}
\begin{document}
	
	\begin{titlepage}
	\begin{center}
		\large 	Московский физико-технический институт \\
		(национальный исследовательский университет) \\
		%Факультет общей и прикладной физики \\
		\vspace{0.2cm}
		
		\vspace{4.5cm}
		Вопрос по выбору \\ \vspace{0.2cm}
		\large (Общая физика) \\ \vspace{0.2cm}
		\LARGE \textbf{Мюонный катализ}
	\end{center}
	\vspace{2.3cm} \large
	
	\begin{center}
		Работу выполнил: \\
		\vspace{10mm}		
		
	\end{center}
	
	\begin{center} \vspace{75mm}
		г. Долгопрудный \\
		2021 год
	\end{center}
\end{titlepage}


\section{Термоядерный синтез}
Реакции ядерного синтеза -- ядерные реакции, при которых лёгкие атомные ядра объединяются в более тяжёлые. Такие реакции могут сопровождаться значительным выделением энергии. Так, в ядерных реакциях синтеза гелия и трития из ядер --- изотопов водорода имеем:
\begin{align}
\mathrm{d} + \mathrm{d} &\longrightarrow \ {}^3\mathrm{He} + \mathrm{n} + 3,3 \ \text{МэВ}\\
\mathrm{d}+\mathrm{d} &\longrightarrow \; \mathrm{t}+\mathrm{p}+4,0  \ \text{МэВ} \\
\mathrm{d}+\mathrm{t} &\longrightarrow \ {}^{4} \mathrm{He}+\mathrm{n}+17,6 \ \text{МэВ} \\
\mathrm{t}+\mathrm{t} &\longrightarrow \ {}^{4} \mathrm{He}+2 \mathrm{n}+11,3 \ \text{МэВ}
\end{align}

\section{Необходимые температуры}

Для протекания ядерных реакций сталкивающиеся ядра должны сблизиться на расстояние порядка $R_{\text {яд }} \simeq 4 \cdot 10^{-13}$ см --- радиуса действия ядерных сил. Этому сближению препятствуют действуюшие между ядрами силы кулоновского (электростатического) отталкивания. Потенциальная энергия этого взаимодействия ядер-изотопов водорода между собой имеет вид
$$
U_{\text {кул }}(r)=\frac{e^{2}}{r}
$$
Высота этого барьера составляет

\[ U_{\max }=\frac{e^{2}}{R_{\text {Яд }}} \simeq 1 \ \text{МэВ} \]

что соответствует средней энергии теплового движения частиц при $T \sim 10^{10} K$

Термоядерный синтез может идти и при более низких температурах за счёт подбарьерного туннелирования. Проницаемость кулоновского барьера (вероятность туннелирования при однократном столкновении) оценим по формуле:
\[
\begin{aligned}
D \approx & \exp \left(-\frac{2}{\hbar} \int_{R_{\text {яд }}}^{R_{\text {ост }}} \sqrt{2 m(U(x)-E)} d x\right) \approx \\
& \approx \exp \left(-\frac{2 e \sqrt{2 m}}{\hbar} \int_{R_{\text {sд }}}^{R_{\text {ост }}} \frac{d x}{\sqrt{x}}\right)=\\
=& \exp \left(-\frac{4 e \sqrt{2 m}}{\hbar}\left(\sqrt{R_{\text {ост }}}-\sqrt{R_{\text {яд }}}\right)\right)
\end{aligned}
\]
где $R_{\text{ост}}=e^{2} / E-$ точка остановки. \\
Таким образом
$$
D(v) \approx \exp \left(-\frac{8 e^{2}}{\hbar v}\right)
$$

\section{Необходимые температуры-2}

Более точный расчёт с учётом максвелловского распределения частиц по скоростям (выражение для проницаемости барьера следует усреднить с учетом распределения Максвелла: $\left.d w=\left(\frac{m}{2 \pi k T}\right)^{3 / 2} e^{-\frac{m v^{2}}{2 k T}} 4 \pi v^{2} d v\right)$ даёт среднюю проницаемость

\[D \sim \exp \left(-\frac{20}{E^{1 / 3}[кэВ]}\right) \]

Соответственно, термоядерная реакция может идти уже при темпеpaтypax $\sim 10^{8}$ K

Но и такие температуры труднодостижимы, поэтому предлагались различные способы осуществления ядерного синтеза при меньших температурах, одним из таких способов является мюонный катализ.

\section{Мезоатом}
Мюон -- элементарная частица, лептон, имеющий массу $m_{\mu}=207 m_{e},$ и являющийся нестабильной частицей со временем жизни $\tau \approx 2,2 \cdot 10^{-6} $ с. 

При попадании в смесь изотопов водорода, мюон образует там мезоатомы $\mu \mathrm{d}$ и $\mu \mathrm{t},$ которые, сталкиваясь затем с молекулами $H_{2}, D_{2}$ и $T_{2}$ (а также НD, НТ и $\mathrm{DT}$ ), образуют мезомолекулы $\mathrm{pp} \mu, \mathrm{pd} \mu, \mathrm{pt} \mu, \mathrm{dt} \mu$ и $\mathrm{tt} \mu$ (или, точнее, мезомолекулярные ионы). Поскольку мюон примерно в 200 раз тяжелее электрона, то размеры мезомолекул во столько же раз меньше размеров молекулярных ионов $H_{2}^{+}, D_{2}^{+}$ и т. д., в которых ядра удалены друг от друга в среднем на расстояние в две атомных единицы $\sim 2 a_{0}=2 \hbar^{2} / m_{e} e^{2} \approx 10^{-8}$ см. Таким образом, в мезомолекулярном ионе ядра удалены на расстояние $\sim 2 a_{\mu}=2 \hbar^{2} / m_{\mu} e^{2} \approx 5 \cdot 10^{-11}$ см.

\section{Время реакции}

В таком ионе благодаря его малым размерам достаточно быстро протекает ядерная реакция синтеза. Оценим время протекания ядерной реакции, считая, что при сближении на расстояние $R$ реакция синтеза происходит с вероятностью 1. По порядку величины время протекания реакции равно $\frac{1}{N w},$ где $w-$ вероятность туннелирования через кулоновский барьер при однократном столкновении, а $N=\frac{1}{T_{\text {кол }}}-$ число ударов о барьер в единицу времени, $T_{\text {кол }}$ -- период радиальных колебаний ядер в ионе. Частоту колебаний для двухатомных молекул можно оценить как
\begin{equation}
\omega_{\text {кол }} \sim \omega_{aт} \sqrt{\frac{m_{e}}{m}}, \quad \omega_{ат}=\frac{m_{e} e^{4}}{\hbar^{3}}
\end{equation}
где $m$ --- приведённая масса ядер. 

Заменяя в $m_{e}$ на $m_{\mu},$ находим частоту колебаний ядер в мезомолекулярных ионах, а с ней и
\begin{equation} 
\tau_{\text {реак }} \sim 2 \pi \frac{\hbar^{3}}{m_{\mu} e^{4}} \sqrt{\frac{m}{m_{\mu}}} D_{\text {кул }}^{-1}(v) 
\end{equation}


(не говорить) Важным здесь является выбор значения скорости $v$. Чтобы получить его, учтём, что энергия водородоподобного мюонного атома (иона с зарядом ядра Zе) равна $\varepsilon_{0}=-\frac{m_{\mu}}{m_{e}} \operatorname{Ry} Z^{2} \simeq-2,8 Z^{2}$ кэВ (для самого нижнего уровня, ядро для простоты считаем бесконечно тяжелым). Известно также, что энергия связи мезомолекулярного иона (в основном состоянии) составляет около $300$ эВ, так что энергия мюона в ионе равна $-3.1$ кэВ. Однако, когда ядра сближаются до ядерных расстояний, мюон уже не различает их по отдельности, а "чувствует" лишь их суммарный заряд, $Z=2 .$ При этом энергия мюона составляет $-11.2$ кэВ. Соответственно, кинетическая энергия ядер при их столкновении оказывается равной около $\varepsilon=8$ кэВ. Отсюда следует значение скорости $v=\sqrt{2 \varepsilon / m}$. 

Для dt$\mu$-иона получаем $T_{кол} \simeq 7 \cdot 10^{-17} \text{c}, D_{ кул} \simeq 5 \cdot 10^{-5}$ и $\tau_{\text {реак}} \simeq 10^{-12}$ с. Как видно, время протекания ядерной реакции на шесть порядков меньше времени жизни мюона. Поэтому, как будет показано ниже, оно не сказывается на числе актов ядерной реакции, инициируемых одним мюоном.

\section{"Отравление" катализатора}

После протекания ядерной реакции мюон может быть подхвачен образующимся в реакции заряженным ядром гелия. Оказываясь приэтом связанным в мезоатомный ион $\mu\mathrm{He}$, мюон уже перестает выступать в роли катализатора ядерного синтеза. Именно это обстоятельство сушественно сказывается на эффективности $\mu$-катализа, то есть на числе инициируемых мюоном ядерных реакций. Вероятность $\omega_{s}$ образования мезоатомов $\mu^{4}\mathrm{He}$ и $\mu^{3}\mathrm{He}$ называют коэффициентом прилипания. Оценка вероятности прилипания мюона к ядру гелия дает
\begin{equation} 
w_{s} \simeq\left(1+\frac{m_{\mu} \varepsilon_{H e}}{4 m_{H e} \varepsilon_{\mu}}\right)^{-4}
\end{equation}
Здесь $\varepsilon_{H e}$ -- энергия образуюшегося ядра гелия, в реакции дейтерия с тритием равна $3,52$ МэВ; $\varepsilon_{\mu}$ - энергия связи мюона в мюонном ионе гелия, равна $11,3$ кэВ.

Получаем $w \sim 0.01$. В случае реакции дейтерия с дейтерием из-за более низкого энерговыделения $w \sim 0.1$.

Малость вероятности прилипания мюона к $\alpha$-частице в реакции d+t делает ее наиболее перспективной, и последнее развитие $\mu$-катализа связано с его исследованием в смеси дейтерия и трития.

%это не очень нужно здесь
%Расчёт скорости образования мезомолекулярного иона -- сложная задача. Сделаем одно замечание. Система dt$\mu$ примечательна тем, что в ней есть состояние с оцень низкой энергией связи $\simeq 0.66$ эВ. Поэтому скорость образования именно этой системы значительно больше скорости образования других систем dd$\mu$ и т.д. Поэтому эта стадия каталитического цикла не вносит сушественных ограничений на его эффективность, и число ядерных реакций, инициируемых одним мюоном, ограничивается вероятностью его прилипания и равно около $100$. Экспериментально полученные значения равны $100-150$.

\newpage 
\section{Цикл мюонного катализа}

Цикл мюонного катализа в смеси $\mathrm{D}_{2}+\mathrm{T}_{2}$, можно представить на схеме (рис. 1), где введены обозначения: $\omega_{s}$ --- вероятность прилипания мюона к гелию, $\lambda_f = \frac{1}{\tau_f} $ --- скорость ядерной реакции синтеза в мезомолекуле $d t \mu, \ \lambda_{a}$ --- скорость образования мезоатомов $d \mu$ и $t \mu, \ \lambda_{d t \mu}$ --- скорость образования мезомолекул $d t \mu, \ \lambda_{d t}$ --- скорость перехода мюона от дейтерия к тритию в реакции изотопного обмена.
%лучше не говорить думаю
%рассмотренныей впервые теоретически в 1957 г., сразу после знаменитого опыта Алвареса, 

Скорость образования мезоатомов $\lambda_{a} \approx 3 \cdot 10^{12} \  \text{c}^{-1}$ значительно больше скоростей $\lambda_{d t}$ и $\lambda_{d t \mu} \ \left(\sim 3 \cdot 10^{8} \  \text{c}^{-1}\right),$ поэтому скорость цикла мюонного катализа $\lambda_{c}$ практически не зависит от $\lambda_{a} .$ При указанном соотношении скоростей время цикла катализа $\tau_{c}=\frac{1}{\lambda_{c}}$ складывается главным образом из времени пребывания мюона в основном состоянии атомов $d \mu$ и атомов $t\mu$. Число циклов мюонного катализа осуществляемых одним мюоном за время его жизни обратно пропорционально вероятности ухода мюона из цикла вследствие распада или прилипания к гелию:
\begin{equation}X_{c} \approx\left(\omega_{s}+\frac{\lambda_{0}}{\lambda_{c} \varphi}\right)^{-1}
\end{equation}
где $\lambda_{0}=\frac{1}{\tau_{\mu}}=0,46 \cdot 10^{6} \ \text{c}^{-1}-$ скорость распада мюона. 

Таким образом, $X_{c} \gg 1,$ если одновременно $\omega_{s} \ll 1$ и $\lambda_{0} / \lambda_{c} \varphi \ll 1 .$ 

В табл. 1 представлены значения для различных циклов $\mu-$катализа
с участием изотопов $\mathrm{p}, \mathrm{d}$ и t. Только для мезомолекулы $\mathrm{dt} \mu$ выполнены все условия, обеспечиваюшие большое число циклов мюонного катализа. Достигнутые в данный момент значения $X_{c}$ составляют $\approx 100-150 .$ 

Таким образом, в смеси дейтерия и трития один мюон может катализировать свыше 100 реакций синтеза и освободить при этом $\sim 2$ ГэВ энергии и $X_{c}$ нейтронов. Выделившейся энергии все еще недостаточно, чтобы
покрыть энергетические затраты на рождение на ускорителе отрицательного мюона (5--10 ГэВ), и общий баланс энергии остается отрицательным даже при $X_{c}=100 .$ Чистый мюонный катализ может стать коммерчески выгодным способом производства энергии лишь при $X_{c} \sim 10^{4}$.

\begin{table}[h]
\centering
\caption{Основные характеристики процессов монного катализа}
\begin{tabular}{|c|c|c|c|c|c|}
\hline
Мезомолекулы & $w_s$ & $\lambda_m,$ с^{-1} & $\lambda_f,$ с^{-1} & $\lambda_{ab},$ с^{-1} & Энерговыделение $Q$, МэВ\\ \hline
pd$\mu$ & 0,99 & $5,8 \cdot 10^6$ & $2,6 \cdot 10^5$ & $1,7 \cdot 10^{10}$ & 5,4 \\ 
pt$\mu$ & 0,94 & $6,8 \cdot 10^6$ & $0,7 \cdot 10^5$ & $0,7 \cdot 10^{10}$ & 20 \\ 
dd$\mu$ & 0,12 & $\sim 4 \cdot 10^6$ & $4,3 \cdot 10^8$ & $3,7 \cdot 10^{7}$ & 3,3 \\ 
dt$\mu$ & $0,43 \cdot 10^{-2}$ & $\sim 3 \cdot 10^8$ & $1,2 \cdot 10^{12}$ & $2,8 \cdot 10^{8}$ & 17,6 \\ 
tt$\mu$ & 0,14 & $3 \cdot 10^6$ & $1,5 \cdot 10^7$ & $1,2 \cdot 10^{9}$ & 11,3\\ \hline
\end{tabular}
\end{table}

\section{Гибридный реактор}
% ?хз как назвать но я думаю тут лучше не углубляться а просто показать типа вот можно это все сделать энергетически выгодным. Если описывать как то утонем пиздец.

Чтобы сделать мюонный катализ выгодным, нужно либо снизить затраты энергии на производство мезонов, либо увеличить выход термоядерной энергии на один мезон. Рассмотрим второй путь. Один из способов увеличения выхода энергии состоит в том, чтобы использовать способность нейтронов, рождающихся в термоядерных реакциях, вступать в ядерные реакции с ядрами урана. 

Дело в том, что одним из продуктов ядерной реакции (1а:d+t=..) являются нейтроны с энергией $\epsilon_n \approx 14$ МэВ. Поток таких нейтронов при облучении им оболочки из урана $^{238}$U, окружающей мюон-каталитический реактор, может служить как для получения энергии деления(а) так и для расширенного воспроизводства ядерного топлива – плутония $^{239}$Pu(б):
\begin{equation}
\mathrm{n} + {}^{238}\mathrm{U} \rightarrow
\left[\begin{array}{l}
\xrightarrow{(а)} осколки + 200 МэВ \\
\xrightarrow{(б)} {}^{239}\mathrm{P}
\end{array}\right.
\end{equation}

Как показывают расчеты энергетический выход при этом (на один мюон) более чем в $100$ раз превышает энерговыделение в реакциях синтеза в мезомолекулярных ионах. На основе такого гибридного реактора уже возможно решение энергетических проблем.

\section{Вывод}

В смеси дейтерия и трития один мюон инициирует около $X_c \sim 100$ реакций синтеза, при этом освобождается энергия $\sim 2$ ГэВ и $X_c$ нейтронов. Затраты на рождение одного мюона $\sim 5$ ГэВ превосходят выделившуюся энергию. Чистый мюонный катализ может стать энергетически выгодным способом получения энергии при $X_c \sim 10^4$. В качестве решения нужно либо снизить затраты энергии на производство мезонов, либо увеличить выход термоядерной энергии на один мезон, например, используя гибридные реакторы.

\end{document}
